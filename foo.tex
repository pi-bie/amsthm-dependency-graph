\documentclass{article}
\usepackage{amsthm}

\theoremstyle{definition}
\newtheorem{defn}{Definition}

\theoremstyle{plain}
\newtheorem{theorem}[defn]{Theorem}
\newtheorem{proposition}[defn]{Proposition}
\newtheorem{lemma}[defn]{Lemma}
\newtheorem{corollary}[defn]{Corollary}

\theoremstyle{remark}
\newtheorem{example}[defn]{Example}
\newtheorem*{remark}{Remark}

\begin{document}
\begin{lemma} \label{lemma:zero}
This one doesn't even need a proof.
\end{lemma}

\begin{lemma} \label{lemma:first}
Some nonsense
\[a^2 = b^2+c^2\]
\begin{proof}
This is self-contained
\end{proof}
\end{lemma}

\begin{proposition} \label{prop:second}
Some result.
\begin{proof}
We invoke Lemma~\ref{lemma:zero}.
\end{proof}
\end{proposition}

\begin{theorem} \label{thm:main}
We do know something.
\begin{proof}
We want to prove stuff. We rely on Lemma~\ref{lemma:first} and also on Proposition~\ref{prop:second} % but not on \ref{lemma:zero}
%Do we really need \ref{lemma:zero}?
We are not sure if we should use Proposition~\ref{prop:nowhere}.
But certainly, we do need Lemma~\ref{lemma:first}.
\end{proof}
\end{theorem}
\end{document}
